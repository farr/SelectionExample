\documentclass[modern]{aastex62}

\usepackage{amsmath}
\usepackage{amssymb}

% Derivatives
\newcommand{\dd}{\mathrm{d}}
\newcommand{\diff}[2]{\frac{\dd #1}{\dd #2}}

% Abbreviated Variables
\newcommand{\Ndet}{N_\mathrm{det}}
\newcommand{\Nndet}{N_\mathrm{ndet}}
\newcommand{\Nnobs}{N_\mathrm{nobs}}
\newcommand{\Nobs}{N_\mathrm{obs}}
\newcommand{\Ntotal}{N_\mathrm{total}}

% Vector shortcuts
\newcommand{\vd}{\vec{d}}
\newcommand{\vlambda}{\vec{\lambda}}
\newcommand{\vtheta}{\vec{\theta}}

\begin{document}
\title{Incorporating Selection Effects in Population Analyses}

\author[0000-0003-1540-8562]{Will M. Farr}
\email{will.farr@stonybrook.edu}
\affiliation{Department of Physics and Astronomy, Stony Brook University, Stony Brook NY 11794, USA}
\affiliation{Center for Computational Astronomy, Flatiron Institute, 162 5th Ave., New York NY 10010, USA}

\author[0000-0002-6134-8946]{Ilya Mandel}
\email{imandel@star.sr.bham.ac.uk}

\author{Jonathan R. Gair}
\email{J.Gair@ed.ac.uk}

\section*{}

Consider a population of objects, each described by some set of parameters
$\vtheta$, that follows a number density $\diff{N}{\vtheta}$.  Let the number
density be parameterized by parameters, $\vlambda$.  For each object in the
population we make a noisy measurement of $\vtheta$, represented by a likelihood
function relating the measured data, $\vd$, to the parameters of the object,
$\vtheta$: $p\left( \vd \mid \vtheta \right)$.  If we have observed a
representative sample (i.e.\ a ``fair draw''), then the appropriate
(unnormalised) joint distribution for the parameters $\left\{ \vtheta_i
\right\}_{i=1}^{\Ntotal}$ and observations $\left\{ \vd_i \right\}$ of the $i =
1, \ldots, \Ntotal$ objects given the parameters $\vlambda$ describing the
population
%
\begin{equation}
  \pi\left(\left\{ \vtheta_i \right\}, \left\{ d_i \right\} \mid \vlambda \right) \propto \left[ \prod_{i=1}^{\Ntotal} p\left( \vd_i \mid \vtheta_i \right) \diff{N}{\vtheta_i}\left( \vlambda \right) \right] \exp\left[ - N\left( \vlambda \right) \right],
\end{equation}
%
where
%
\begin{equation}
N\left( \vlambda \right) \equiv \int \dd \vd \, \dd \vtheta p\left( \vd \mid \vtheta \right) \diff{N}{\vtheta}\left( \lambda \right)
\end{equation}
%
is the expected number of objects in the population\footnote{The rationale for
writing this as a double-integral, when the integral over $\vd$ is in fact
trivial---since the likelihood is normalised over $\vd$---will become apparent
below.}.  This is the standard posterior for a hierarchical analysis of an
inhomogeneous Poisson process
\citep{Loredo1995,Hogg2010,Mandel2010,Youdin2011,Foreman-Mackey2014,Farr2015,Barrett2018}.

Now suppose that, based on the observed data, some objects are classed as
``observable'' and others are ``un-observable.''  For example, a survey may
impose a per-pixel or per-apeture threshold on the flux for inclusion of
point-sources in a catalog, or a gravitational wave detector my only report
events whose signal-to-noise ratio rises above some predetermined threshold,
or\ldots.  The key point here is that the selection of ``observable'' objects is
made by examining the data, $\vd_i$, for each object; this is by far the most
common case for astronomical observations.  Then the complete set of
observations partitions into two subsets:
%
\begin{multline}
  \pi\left(\left\{ \vtheta_i \right\}, \left\{ \vtheta_j \right\}, \left\{ d_i \right\}, \left\{ d_j \right\} \mid \vlambda \right) \propto \left[ \prod_{i=1}^{\Nobs} p\left( \vd_i \mid \vtheta_i \right) \diff{N}{\vtheta_i}\left( \vlambda \right) \right] \\ \times \left[ \prod_{j=1}^{\Nnobs} p\left( \vd_j \mid \vtheta_j \right) \diff{N}{\vtheta_j}\left( \vlambda \right) \right] \exp\left[ - N\left( \vlambda \right) \right].
\end{multline}
%
Again, a key point is that we can perform this partitioning simply by examining
the \emph{data} obtained for each object.

It is common for the data associated with ``non-observable'' objects to be
completely \emph{censored}; that is, it often does not appear in a catalog or
otherwise at all.  In this case, it is appropriate to marginalize over the
parameters and (unknown) data for the ``non-observable'' objects.  Doing so
destroys the distinguishability inherent in the inhomogeneous Poisson
distribution, so we must introduce a factor of $\Nnobs!$ to account for the
over-counting:
%
\begin{equation}
  \pi\left(\left\{ \vtheta_i \right\}, \left\{ d_i \right\} \mid \vlambda \right) \propto \left[ \prod_{i=1}^{\Nobs} p\left( \vd_i \mid \vtheta_i \right) \diff{N}{\vtheta_i}\left( \vlambda \right) \right] \frac{\Nndet^{\Nnobs}\left( \vlambda \right)}{\Nnobs!} \exp\left[ - N\left( \vlambda \right) \right],
\end{equation}
%
where
%
\begin{equation}
\Nndet\left( \vlambda \right) \equiv \int_{\left\{ \vd \mid \textnormal{non-detection} \right\}} \dd \vd \, \dd \vtheta \, p\left( \vd \mid \vtheta \right) \diff{N}{\vtheta}\left( \vlambda \right)
\end{equation}
%
is the expected number of non-detections in the population model.  It is further
common to not even know \emph{how many} non-detected objects there were in a
given survey or data set.  In this case we must marginalize---sum, since
counting is a discrete operation---over the unknown number of non-detections,
$\Nnobs$, yielding
%
\begin{equation}
\pi\left(\left\{ \vtheta_i \right\}, \left\{ d_i \right\} \mid \vlambda \right) \propto \left[ \prod_{i=1}^{\Nobs} p\left( \vd_i \mid \vtheta_i \right) \diff{N}{\vtheta_i}\left( \vlambda \right) \right] \exp\left[ - \left( N\left( \vlambda \right) - \Nndet\left( \vlambda \right) \right) \right],
\end{equation}
%
or
%
\begin{equation}
  \pi\left(\left\{ \vtheta_i \right\}, \left\{ d_i \right\} \mid \vlambda \right) \propto \left[ \prod_{i=1}^{\Nobs} p\left( \vd_i \mid \vtheta_i \right) \diff{N}{\vtheta_i}\left( \vlambda \right) \right] \exp\left[ - \Ndet\left( \vlambda \right) \right],
\end{equation}
%
where $\Ndet$---the compliment of $\Nndet$---is the expected number of
detections under the population model:
%
\begin{equation}
  \Ndet\left( \vlambda \right) \equiv \int_{\left\{ \vd \mid \textnormal{detection} \right\}} \dd \vd \, \dd \vtheta \, p\left( \vd \mid \vtheta \right) \diff{N}{\vtheta}\left( \vlambda \right).
\end{equation}
%
This equation is the posterior for a hierarchical analysis of the number density
and properties of objects from a data set subject to selection effects
\citep[e.g.][]{Gair2010,Youdin2011,Fishbach2018,Wysocki2018}.

If we re-parameterize $\diff{N}{\vtheta}$ so that we can write
%
\begin{equation}
  \diff{N}{\vtheta} \equiv \Lambda p\left( \vtheta \mid \vlambda{}' \right)
\end{equation}
%
with $p\left( \vtheta \mid \vlambda{}'\right)$ integrating to 1 over the
population for any value of the new parameters $\vlambda{}'$, impose a prior
$p\left( \Lambda \right)\propto 1/\Lambda$, and marginalize over $\Lambda$, we
arrive at the treatment of selection functions for estimating population
distributions from \citet{Loredo2004,O1-BBH,Mandel2016}, as noted by
\citet{Fishbach2018}.

Note that the commonly-employed technique of modifying $\diff{N}{\vtheta}$ to
account for the selection function is not correct, and will lead to biased
results as long as the selection is dependent only on the observed
data\footnote{An example where the selection may be parameter- rather than
data-dependent is in surveys of objects that have been selected based on data in
yet other surveys.  For example, X-ray selected populations of galaxy clusters
in a weak-lensing catalog.}.

\acknowledgments

We thank Maggie Lieu for pointing out the counterexample given in the final
footnote of this document.  An example of a worked population analysis using
this method can be found at \url{https://github.com/farr/SelectionExample}.
IM's work was performed in part at Aspen Center for Physics, which is supported
by National Science Foundation grant PHY-1607611; IM's visit there was partially
supported by a grant from the Simons Foundation.

\newpage

\bibliography{selection}

\end{document}
